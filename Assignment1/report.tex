\documentclass{article}
\usepackage{graphicx}
\usepackage{geometry}
\usepackage{listings}
\usepackage{color}
\definecolor{dkgreen}{rgb}{0,0.6,0}
\definecolor{gray}{rgb}{0.5,0.5,0.5}
\definecolor{mauve}{rgb}{0.58,0,0.82}

\lstset{frame=tb,
   language=Java,
   aboveskip=3mm,
   belowskip=3mm,
   showstringspaces=false,
   columns=flexible,
   basicstyle={\small\ttfamily},
   numbers=none,
   numberstyle=\tiny\color{gray},
   keywordstyle=\color{blue},
   commentstyle=\color{gray},
   stringstyle=\color{mauve},
   breaklines=true,
   breakatwhitespace=true,
   tabsize=4
}
\geometry{
  top=20mm,
}

\title{COL703 - Assignment1\\Scanning And Parsing}
\author{Suyash Agrawal\\2015CS10262}

\begin{document}
\maketitle

\section{Lexer}
Lexer parsed the input into the following tokens:
\begin{itemize}
  \item NOT
  \item OR
  \item AND
  \item IF
  \item THEN
  \item IFF
  \item ELSE
  \item TRUE
  \item FALSE
  \item LPAR (denotes `(')
  \item RPAR (denotes `)')
  \item ATOM (denotes string)
  \item EOL (denotes end of line)
\end{itemize}

\section{Parser}
In order to correctly parse the grammar without using high level directives like "\%left"
, I introduced additional non-terminals in my grammar. The Terminals were:\\
\begin{lstlisting}
P_NOT, P_AND, P_OR, P_IF, P_THEN, P_IFF, P_ELSE, P_TRUE
P_FALSE, P_RPAR, P_LPAR, P_ATOM, P_ILLCH, P_EOF, P_EOL
\end{lstlisting}
The Non-Terminals were:\\
\begin{lstlisting}
propListR,propR,iffR,iteR,orR,andR,negR,basicR,wordR
\end{lstlisting}
My grammar is as follows:
\begin{lstlisting}
propListR: propR P_EOL propListR            (propR::propListR)
    | propR                                 ([propR])

propR: P_IF propR P_THEN propR              (IMP(propR1,propR2))
    | iffR                                  (iffR)

iffR: iteR P_IFF propR                      (IFF(iteR,propR))
    | iteR                                  (iteR)

iteR: P_IF propR P_THEN iffR P_ELSE propR   (ITE(propR1,iffR,propR2))
    | orR                                   (orR)

orR: orR P_OR andR                          (OR(orR,andR))
    | andR                                  (andR)

andR: andR P_AND negR                       (AND(andR,negR))
    | negR                                  (negR)

negR: P_NOT negR                            (NOT(negR))
    | basicR                                (basicR)

basicR: P_TRUE                              (TOP)
    | P_FALSE                               (BOTTOM)
    | P_LPAR propR P_RPAR                   (propR)
    | wordR                                 (ATOM(wordR))

wordR: P_ATOM wordR                         ((P_ATOM)^" "^(wordR))
    | P_ATOM                                (P_ATOM)
\end{lstlisting}
The basic intuition was to first club the tokens around the high priority operators like AND OR etc.
and then use these non-terminal in the grammar of other tokens.
Also, ordering of these non-terminal were used to generate right and left associativity of the operators.\\
\\
These were the essence of my grammar:
\begin{itemize}
  \item Ground terms and parenthesized expressions are given highest preference.
  \item Priority of NOT , AND , OR are implemented using hierarchical structure of grammar
  \item No unparenthesised "IF THEN" expression is allowed to occur between "THEN" and "ELSE" of
  "IF THEN ELSE" operator.
  \item IFF is made right associative by constraining the leftmost IFF to contain expression of higher
  precedence of IFF.
\end{itemize}

\section{Running}
\begin{lstlisting}{bash}
  sml wrapper.sml <input_file> <output_file>
\end{lstlisting}

NOTE:
\begin{itemize}
\item The input file should not contain `.' at end of statements.
\item The output file should exist. i.e. Program will not automatically create the file.
\end{itemize}


\end{document}