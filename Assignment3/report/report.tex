\documentclass{article}
\usepackage{graphicx}
\usepackage{geometry}
\usepackage{listings}
\usepackage{color}
\definecolor{dkgreen}{rgb}{0,0.6,0}
\definecolor{gray}{rgb}{0.5,0.5,0.5}
\definecolor{mauve}{rgb}{0.58,0,0.82}

\lstset{frame=tb,
   language=Java,
   aboveskip=3mm,
   belowskip=3mm,
   showstringspaces=false,
   columns=flexible,
   basicstyle={\small\ttfamily},
   numbers=none,
   numberstyle=\tiny\color{gray},
   keywordstyle=\color{blue},
   commentstyle=\color{gray},
   stringstyle=\color{mauve},
   breaklines=true,
   breakatwhitespace=true,
   tabsize=4
}
\geometry{
  top=20mm,
}

\title{COL703 - Assignment3\\First Order Resolution}
\author{Suyash Agrawal\\2015CS10262}

\begin{document}
\maketitle

\section*{Introduction}
The objective of this assigment was to check satisfiability of first order logic statements
using Resolution Refutation method.

\section*{Procedure}
The following procedure was used for resolution:
\begin{itemize}
    \item First standardize the variables of the formula apart to prevent free variable capture during various steps.
    \item Then convert the standardized formula to Prenex Normal form.
    \item Convert the quantifier free part to Conjunctive Normal Form.
    \item Skolemize the formula to get rid of EXISTS quantifier.
    \item Convert the formula to set notation, i.e. formula is a set of clauses and each clause is a set of predicate.
    \item Resolution Step:
    \begin{itemize}
        \item Select two clauses from set such that there is a predicate p in one of them and NOT(predicate p) in other
              and they are unifiable.
        \item Compute the most general unifier for these two predicate.
        \item Apply the m.g.u. computed on the rest of the predicates in the selected clauses.
        \item Remove these two clauses from the original set of clauses and add the union of the predicates in these 
              clauses after removing the selected predicates and applying the m.g.u.
    \end{itemize}
    \item Do resolution step repeatedly till it is possible to find two unifiable perdicates.
    \item The result is:
    \begin{itemize}
        \item If Empty clause is derived then False (as contradiction is shown)
        \item Else True.
    \end{itemize}
\end{itemize}

\section*{NOTE}
Since the resolution step is itself indeterministic (as it does not specify the procedure for choosing which
terms to unify), this program can give false positives i.e. it can output true for unsatisfiable formula
,as for some order of resolution we may not be able to derive contradiction.\\
Though, the program will never give false negative, i.e. if it outputs false for a formula then that
formula is definitely false as we are able to derive contradiction from it.

\section*{Running}
Please see ``test.sml'' attached to see how to use the program. It also contains sample testcases.


\end{document}
